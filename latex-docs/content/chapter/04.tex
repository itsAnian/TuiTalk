\chapter{Schwierigkeiten und alternative Lösungen}

% Redis cluster
\section{Synchronisation der Nachrichten}
\subsection{Schwierigkeiten}
Zu beginn des Entwicklungsprozess wurde nur ein Backend verwendet was die Kommunikation zwischen den Clients und dem Backend vereinfachte.
Allerdings hat sich hieraus das Problem der Sychronisation der Nachrichten ergeben.
Dafür wurde ein Redis-Cluster aufgesetzt, welcher die Kommunikation zwischen den Backendinstanzen ermöglicht.
Hierbei wurde initial ein Redis-Cluster mit drei Nodes aufgesetzt.
Da das Cluster bei den Ausfall einer Node abgestürzt ist musste auch hier für Redundanz gesorgt werden.
Deshalb ist als Folgemaßnahme ein erweitertes Redis-Cluster in Betrieb genommen worden.
Dieses neue Cluster besitzt sechs Nodes, drei davon sind Master-Nodes und die restlichen drei sind zu den Master-Nodes die jeweiligen Replicas.

\subsection{Alternative Lösungen}
Cassandra?
