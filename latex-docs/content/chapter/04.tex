\chapter{Schwierigkeiten und alternative Lösungen}

% Redis cluster
\section{Synchronisation der Nachrichten}
\subsection{Schwierigkeiten}
Zu beginn des Entwicklungsprozess wurde nur ein Backend verwendet was die Kommunikation zwischen den Clients und dem Backend vereinfachte.
Allerdings hat sich hieraus das Problem der Sychronisation der Nachrichten ergeben.
Dafür wurde ein Redis-Cluster aufgesetzt, welcher die Kommunikation zwischen den Backendinstanzen ermöglicht.
Hierbei wurde initial ein Redis-Cluster mit drei Nodes aufgesetzt.
Da das Cluster bei den Ausfall einer Node abgestürzt ist musste für Redundanz gesorgt werden, weshalb als Folgemaßnahme ein erweitertes Redis-Cluster in Betrieb genommen wurde.
Dieses neue Cluster besitzt sechs Nodes, drei davon sind Master-Nodes und die restlichen drei sind zu den Master-Nodes die jeweiligen Replicas.

\subsection{Alternative Lösungen}
Alternativ kann ein Eventbasiertets System verwendet werden.
Dafür kann Kafka als Messaging Queue verwendet werden.
Die Nachrichten werden dann als Event in die Kafka Queue hinzugefügt.
Die Backendinstanzen lesen die Nachrichten aus der Queue und verarbeiten diese.
Damit die Nachrichten persistiert werden, könnte weiterhin eine PostgreSQL-Datenbank verwendet werden.


\section{Datenbank}

\subsection{Alternative Lösungen}
Die Datenbank ist in diesem System ein Single Point of Failure, das System kann während des laufenden Betriebes mit einem Ausfall umgehen.
Auf einen längeren Zeitraum ist die Ausfallsicherheit jedoch nicht gegeben, dafür müsste eine verteilte Datenbank verwendet werden. 
Hierfür wäre eine Cassandra Datenbank geeignet.
