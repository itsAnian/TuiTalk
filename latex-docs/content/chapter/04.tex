\chapter{Reflexion}
\section{Rust}
Da vor diesem Projekt fast keine Vorkenntnisse zur Programmierung in Rust bestanden, war der Einstieg für dieses Projekt sehr aufwendig. Aufgrund dieser Einstiegshürde befinden sich in diesem Projekt Altlasten, welche in Zukunft für die weiterarbeit an diesem Projekt überarbeitet werden müssen, allerdings den Rahmen dieses Projekts gesprengt hätten.

\section{Talkprotokoll}
Initial war das Talkprotokoll auf einen Nachrichtentyp begrenzt, welcher wiefolgt aussah:

\begin{lstlisting}[caption=Altes Kommunikationsprotokoll, label=Altes_Kommunikationsprotokoll, basicstyle=\ttfamily\scriptsize]
pub struct TalkMessage {
    pub uuid: Uuid,
    pub username: String,
    pub text: String,
    pub room_id: i32,
    pub unixtime: u64
}
\end{lstlisting}
Dieses wurde für unter anderem für die Anzeige von Errors und Informationen wie Namensänderungen genutzt durch setzen des Namens zu Error oder Info und ähnlichem. Durch die Überarbeitung des Protokolls wurde es einfacher weitere Nachrichtentypen zu implementieren, wie zum Beispiel die Implementierung des Fetch Befehls.

\section{Wasm-Client}
Aufgrund der bereits genannten Einstiegshürde und da zu Beginn des Projekts mit dem Wasm-Client begonnen wurde und es initial nicht so funktioniert hat wie erwartet, hat sich diese Einstellung generell auf WebAssembly übertragen. Dies hat zu einer großen Bevorzugung des Tui-Client geführt, weshalb der Wasm-Client vernachlässigt wurde und nicht sein volles Potenzial entfalten konnte.
