%!TEX root = ../../main.tex

\chapter{Umsetzung}
Für die Kommunikation zwischen den Clients und dem Backend wurde ein einheitliches Protokoll definiert. Dieses beinhaltet Nachrichten, welche von den Clients zum Backend kommen, Nachrichten welche vom Backend zurück an die Clients gehen und Nachrichten, welche in beide Richtungen geschickt werden. Durch dieses Protokoll ist die Erweiterung für andere Nachrichtentypen ohne großen Aufwand möglich.

\begin{lstlisting}[caption=Kommunikationsprotokoll, label=Kommunikationsprotokoll, basicstyle=\ttfamily\scriptsize]
pub enum TalkProtocol {
    // Client -> Server Commands
    JoinRoom { room_id: i32, uuid: Uuid, username: String, unixtime: u64},
    LeaveRoom { room_id: i32, uuid: Uuid, username: String, unixtime: u64},
    ChangeName {uuid: Uuid, username: String, old_username: String, unixtime: u64},
    Fetch { room_id: i32, limit: i64, fetch_before: u64},
    LocalError { message: String },
    LocalInformation { message: String },

    // Server -> Client Events
    UserJoined { uuid: Uuid, username: String, room_id: i32, unixtime: u64 },
    UserLeft { uuid: Uuid, username: String, room_id: i32, unixtime: u64  },
    UsernameChanged {uuid: Uuid, username: String, old_username: String, unixtime: u64},
    History { text: Vec<TalkProtocol> },
    Error { code: String, message: String },


    // Server <-> Client
    PostMessage { message: TalkMessage },
}
\end{lstlisting}

\section{Clients}
Dieses Protokoll befindet sich neben dem Verbindungsaufbau in einer geteilten Bibliothek. Da die \ac{WS} Verbindung für die unterschiedlichen Clients auf unterschiedlichen Bibliotheken beruht mussten diese seperat implementiert werden. Um die Verbindungen trotzdem gesammelt haben wurden diese auch in die geteilte Bibliothek eingebunden.

\subsection{Funktionalitäten}
Beide Clients bieten neben den Grundfunktionen wie Senden und Empfangen von Nachrichten, die folgenden Funktionen:

\begin{table}[h!]
\begin{center}
\begin{tabular}{ | l | m{10cm} | }
\hline
\textbf{Funktion} & \textbf{Beschreibung} \\ 
\hline
/help & Zeigt Informationen zu den verfügbaren Befehlen \\ 
\hline
/clear & Löscht den Chatverlauf \\ 
\hline
/name <String> & Setzt den Nutzernamen \\ 
\hline
/room <Integer> & Wechselt den Raum \\ 
\hline
/fetch <Integer> & Holt ausgehend von der ersten Nachricht im Chatverlauf die vorherigen Nachrichten \\ 
\hline
\end{tabular}
\end{center}
\caption{Funktionen der Clients}
\label{tab:clientcommands}
\end{table}

\section{Backend}
